\documentclass{article}

\usepackage{geometry}
\usepackage{hyperref}

\title {Khidmat: Bagh e Sakina}

\author{
  Ali Asghar Yousuf\\ ay06993
  \and
  Muhammad Azeem Haider\\ mh06860
  \and
  Mohammad Shahid Mahmood\\ mm06600
  \and
  Syed Ibrahim Ali Haider\\ sh06869
}
\date{\today}  

\begin{document}
\maketitle

% Use first person plural (we, us) even if you did the Khidmat individually.

% An introduction of the project, no more than 2 sentences. Provide the highest level of detail only. Other details will come later.
% Typically, "This project is to <short description of porject> for/at <client>."
The objective of this project is to develop an application that will digitalize
an activity designed to enhance student engagement through interactive learning
experiences for Bagh e Sakina.
% About the client.
Bagh e Sakina is a non-profit organization dedicated to developing and
implementing initiatives that uplift and empower children, providing them with
the resources and support needed to ensure a brighter future. Their vision
emphasizes that every child deserves fundamental rights and opportunities
crucial for their well-being.

% About the project.
Bagh e Sakina has an activity book that is used to engage with children. The
activity book is a collection of activities that are designed to enhance
student engagement through interactive learning experiences. The activities are
designed to be fun and engaging, and they are used to know the students better.
The purpose of this project is to develop an application that will digitalize
the activity book and make it more accessible virtually. This will allow the
students to engage with the activities in a more interactive and engaging way.

% About the plan of work.
We will work part time remotely under the supervision of the Bagh e Sakina
team. The goal is to develop, test, and deploy the system by the end of our
Khidmat.

% Copy-paste this section with necessary modifcations for each week.
\newpage % Start the report for each week on a new page.
\section*{Week 1: 9--14 July, 2018}

% A summary, maximum 2 sentences, of this week's activities.
We spent this week meeting several stakeholders in order to understand the
shortcomings of ACCUPLACER and the expectations from the new system.

\begin{tabular}{|l|l|l|l|}
  \hline
  Item     & Activity            & Time     & ID       \\\hline\hline
  1        & Met Admissions team & 3 hrs    & st1      \\\hline
  2        & Met faculty         & 2 hrs    & st2      \\\hline
  3        & Met IT team         & 3 hrs    & st1, st2 \\\hline
  $\vdots$ & $\vdots$            & $\vdots$ & $\vdots$ \\\hline
\end{tabular}

The total time spent on the Khidmat this week is as follows.

\begin{tabular}{|l|l|}
  \hline
  ID  & Total Hours \\\hline\hline
  st1 & 7 hours     \\\hline
  st2 & 6 hours     \\\hline
\end{tabular}

% Other weeks ...

\newpage
\section*{Conclusion}

% Remind the reader about the project. Summarise your activities over the course of the project.
Our project was to build a new testing system for Habib University to replace
ACCUPLACER for its entrance examination. We started by meeting all the
stakeholders to understand their expectations from the new system. We then
identified the necessary tools to build the required system and trained
ourselves on them. Development and testing were carried out in collaboration
with the IT team so that any shortcomings were identified and catered to as we
went along. The system was then deployed and officers from the Admissions Team
were trained to use it.

\newpage
\thispagestyle{empty}
% Show your external supervisor your report, especially the weekly upates; have them sign a printed copy of this page; scan the signed page; and include the scanned page in this document as an image.

\begin{center}
  {\Large\bf Khidmat Completion Form}\\[5pt]
  \small To be completed by the external supervisor.
\end{center}
\bigskip

\noindent{\it Please use the space below to provide any comments you may have on the students' performance, the Khidmat program, or any other feedback you want to share with Habib University's Khidmat committee. We can also be reached at \href{mailto:khidmat@sse.habib.edu.pk}{khidmat@sse.habib.edu.pk}.}
\vfill

\begin{center}
  \rule{.8\textwidth}{.5pt}
\end{center}
\medskip

% Insert your name below.

I hereby certify that I supervised Ali Asghar Yousuf, Muhammad Azeem Haider,
Mohammad Shahid Mahmood and Syed Ibrahim Ali Haider for the Khidmat described
in this report. Furthermore, that I have read and agree with the weekly updates
included in this report. My signature below marks the successful completion of
the work undertaken for the Khidmat.\\ \bigskip \bigskip

\noindent\begin{tabular}{@{}p{.6\textwidth}@{\hspace{.1\textwidth}}p{.3\textwidth}}
  \hrulefill         & \hrulefill        \\
  Name and signature & Location and date
\end{tabular}

\end{document}
